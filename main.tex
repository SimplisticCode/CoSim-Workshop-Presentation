\documentclass{beamer}
\usepackage[utf8]{inputenc}

\usepackage{fourier} %font utopia imported
\usetheme{Madrid}
\usecolortheme{default}

%------------------------------------------------------------
%This block of code defines the information to appear in the
%Title page
\title[FMI-Based initialization plugin for Maestro 2] %optional
{An FMI-Based initialization plugin for INTO-CPS Maestro 2}
\date{\today}
\usepackage{booktabs}
\usepackage{natbib}
\usepackage{tikz}
\usepackage{multicol}
\usepackage{hyperref}
\usepackage{pgfplots}
\usepackage{listings}
\usepackage{tikz}
\usepackage{environ}
\usepackage{filecontents}
\usepackage{verbatim}
\usepackage{graphicx}
\usepackage{amsfonts}
\usepackage{amsmath}
\usepackage{amssymb}
\usepackage{stmaryrd}
\usepackage{amstext}
\usepackage{semantic}

\definecolor{codegreen}{rgb}{0,0.6,0}
\definecolor{codegray}{rgb}{0.5,0.5,0.5}
\definecolor{codepurple}{rgb}{0.58,0,0.82}
\definecolor{backcolour}{rgb}{0.95,0.95,0.92}
\usefonttheme{serif} % <----------- HERE

\lstdefinestyle{mystyle}{
    backgroundcolor=\color{backcolour},   
    commentstyle=\color{codegreen},
    keywordstyle=\color{magenta},
    numberstyle=\tiny\color{codegray},
    stringstyle=\color{codepurple},
    basicstyle=\ttfamily\footnotesize,
    breakatwhitespace=false,         
    breaklines=true,                 
    captionpos=b,                    
    keepspaces=true,                 
    numbers=left,                    
    numbersep=5pt,                  
    showspaces=false,                
    showstringspaces=false,
    showtabs=false,                  
    tabsize=2
}

\lstset{style=mystyle}
\begin{filecontents*}{incorrect.csv}
a,b,c
0,0,0
0.008,-0.000313843,-0.0313843
0.016,-0.00125348,-0.125348
0.024,-0.00280998,-0.280998
0.032,-0.00496389,-0.496389
0.04,-0.00768499,-0.768499
0.048,-0.0109346,-1.09346
0.056,-0.014669,-1.4669
0.064,-0.0188426,-1.88426
0.072,-0.0234109,-2.34109
0.08,-0.0283315,-2.83315
0.088,-0.0335644,-3.35644
0.096,-0.0390719,-3.90719
0.104,-0.044817,-4.4817
0.112,-0.0507637,-5.07637
0.12,-0.0568755,-5.68755
0.128,-0.0631156,-6.31156
0.136,-0.0694474,-6.94474
0.144,-0.0758344,-7.58344
0.152,-0.0822408,-8.22408
0.16,-0.0886322,-8.86322
0.168,-0.0949758,-9.49758
0.176,-0.10124,-10.124
0.184,-0.107397,-10.7397
0.192,-0.113418,-11.3418
0.2,-0.119277,-11.9277
0.208,-0.124952,-12.4952
0.216,-0.130421,-13.0421
0.224,-0.135663,-13.5663
0.232,-0.140659,-14.0659
0.24,-0.145395,-14.5395
0.248,-0.149855,-14.9855
0.256,-0.154027,-15.4027
0.264,-0.157901,-15.7901
0.272,-0.161467,-16.1467
0.28,-0.164721,-16.4721
0.288,-0.167656,-16.7656
0.296,-0.170271,-17.0271
0.304,-0.172564,-17.2564
0.312,-0.174535,-17.4535
0.32,-0.176188,-17.6188
0.328,-0.177525,-17.7525
0.336,-0.178553,-17.8553
0.344,-0.179277,-17.9277
0.352,-0.179705,-17.9705
0.36,-0.179848,-17.9848
0.368,-0.179715,-17.9715
0.376,-0.179317,-17.9317
0.384,-0.178666,-17.8666
0.392,-0.177777,-17.7777
0.4,-0.176662,-17.6662
0.408,-0.175336,-17.5336
0.416,-0.173814,-17.3814
0.424,-0.172111,-17.2111
0.432,-0.170243,-17.0243
0.44,-0.168226,-16.8226
0.448,-0.166076,-16.6076
0.456,-0.16381,-16.381
0.464,-0.161443,-16.1443
0.472,-0.158991,-15.8991
0.48,-0.15647,-15.647
0.488,-0.153897,-15.3897
0.496,-0.151286,-15.1286
0.504,-0.148651,-14.8651
0.512,-0.146009,-14.6009
0.52,-0.143372,-14.3372
0.528,-0.140753,-14.0753
0.536,-0.138166,-13.8166
0.544,-0.135623,-13.5623
0.552,-0.133134,-13.3134
0.56,-0.130712,-13.0712
0.568,-0.128365,-12.8365
0.576,-0.126102,-12.6102
0.584,-0.123933,-12.3933
0.592,-0.121865,-12.1865
0.6,-0.119904,-11.9904
0.608,-0.118056,-11.8056
0.616,-0.116327,-11.6327
0.624,-0.114721,-11.4721
0.632,-0.113241,-11.3241
0.64,-0.11189,-11.189
0.648,-0.110671,-11.0671
0.656,-0.109583,-10.9583
0.664,-0.108629,-10.8629
0.672,-0.107807,-10.7807
0.68,-0.107117,-10.7117
0.688,-0.106557,-10.6557
0.696,-0.106125,-10.6125
0.704,-0.105818,-10.5818
0.712,-0.105634,-10.5634
0.72,-0.105568,-10.5568
0.728,-0.105616,-10.5616
0.736,-0.105773,-10.5773
0.744,-0.106035,-10.6035
0.752,-0.106396,-10.6396
0.76,-0.106851,-10.6851
0.768,-0.107392,-10.7392
0.776,-0.108016,-10.8016
0.784,-0.108714,-10.8714
0.792,-0.10948,-10.948
0.8,-0.110309,-11.0309
0.808,-0.111193,-11.1193
0.816,-0.112125,-11.2125
0.824,-0.113099,-11.3099
0.832,-0.114109,-11.4109
0.84,-0.115148,-11.5148
0.848,-0.116208,-11.6208
0.856,-0.117285,-11.7285
0.864,-0.118371,-11.8371
0.872,-0.119462,-11.9462
0.88,-0.12055,-12.055
0.888,-0.121632,-12.1632
0.896,-0.1227,-12.27
0.904,-0.123751,-12.3751
0.912,-0.124779,-12.4779
0.92,-0.12578,-12.578
0.928,-0.126751,-12.6751
0.936,-0.127687,-12.7687
0.944,-0.128584,-12.8584
0.952,-0.12944,-12.944
0.96,-0.130252,-13.0252
0.968,-0.131018,-13.1018
0.976,-0.131734,-13.1734
0.984,-0.1324,-13.24
0.992,-0.133014,-13.3014
1,-0.133575,-13.3575
1.008,-0.134082,-13.4082
1.016,-0.134534,-13.4534
1.024,-0.134931,-13.4931
1.032,-0.135274,-13.5274
1.04,-0.135562,-13.5562
1.048,-0.135796,-13.5796
1.056,-0.135978,-13.5978
1.064,-0.136108,-13.6108
1.072,-0.136187,-13.6187
1.08,-0.136217,-13.6217
1.088,-0.1362,-13.62
1.096,-0.136138,-13.6138
1.104,-0.136033,-13.6033
1.112,-0.135886,-13.5886
1.12,-0.135701,-13.5701
1.128,-0.13548,-13.548
1.136,-0.135225,-13.5225
1.144,-0.134939,-13.4939
1.152,-0.134624,-13.4624
1.16,-0.134284,-13.4284
1.168,-0.133921,-13.3921
1.176,-0.133537,-13.3537
1.184,-0.133136,-13.3136
1.192,-0.132721,-13.2721
1.2,-0.132293,-13.2293
1.208,-0.131856,-13.1856
1.216,-0.131412,-13.1412
1.224,-0.130964,-13.0964
1.232,-0.130514,-13.0514
1.24,-0.130065,-13.0065
1.248,-0.129619,-12.9619
1.256,-0.129177,-12.9177
1.264,-0.128743,-12.8743
1.272,-0.128318,-12.8318
1.28,-0.127904,-12.7904
1.288,-0.127503,-12.7503
1.296,-0.127116,-12.7116
1.304,-0.126745,-12.6745
1.312,-0.12639,-12.639
1.32,-0.126054,-12.6054
1.328,-0.125737,-12.5737
1.336,-0.12544,-12.544
1.344,-0.125164,-12.5164
1.352,-0.124909,-12.4909
1.36,-0.124676,-12.4676
1.368,-0.124466,-12.4466
1.376,-0.124278,-12.4278
1.384,-0.124112,-12.4112
1.392,-0.12397,-12.397
1.4,-0.123849,-12.3849
1.408,-0.123751,-12.3751
1.416,-0.123675,-12.3675
1.424,-0.12362,-12.362
1.432,-0.123586,-12.3586
1.44,-0.123572,-12.3572
1.448,-0.123577,-12.3577
1.456,-0.123602,-12.3602
1.464,-0.123644,-12.3644
1.472,-0.123703,-12.3703
1.48,-0.123779,-12.3779
1.488,-0.123869,-12.3869
1.496,-0.123974,-12.3974
1.504,-0.124091,-12.4091
1.512,-0.12422,-12.422
1.52,-0.12436,-12.436
1.528,-0.124509,-12.4509
1.536,-0.124667,-12.4667
1.544,-0.124832,-12.4832
1.552,-0.125003,-12.5003
1.56,-0.125179,-12.5179
1.568,-0.125359,-12.5359
1.576,-0.125542,-12.5542
1.584,-0.125727,-12.5727
1.592,-0.125912,-12.5912
1.6,-0.126098,-12.6098
1.608,-0.126282,-12.6282
1.616,-0.126464,-12.6464
1.624,-0.126644,-12.6644
1.632,-0.126819,-12.6819
1.64,-0.12699,-12.699
1.648,-0.127156,-12.7156
1.656,-0.127316,-12.7316
1.664,-0.12747,-12.747
1.672,-0.127617,-12.7617
1.68,-0.127756,-12.7756
1.688,-0.127887,-12.7887
1.696,-0.128011,-12.8011
1.704,-0.128125,-12.8125
1.712,-0.128231,-12.8231
1.72,-0.128327,-12.8327
1.728,-0.128415,-12.8415
1.736,-0.128493,-12.8493
1.744,-0.128562,-12.8562
1.752,-0.128621,-12.8621
1.76,-0.128672,-12.8672
1.768,-0.128713,-12.8713
1.776,-0.128745,-12.8745
1.784,-0.128768,-12.8768
1.792,-0.128783,-12.8783
1.8,-0.128789,-12.8789
1.808,-0.128787,-12.8787
1.816,-0.128778,-12.8778
1.824,-0.128761,-12.8761
1.832,-0.128736,-12.8736
1.84,-0.128706,-12.8706
1.848,-0.128669,-12.8669
1.856,-0.128626,-12.8626
1.864,-0.128578,-12.8578
1.872,-0.128525,-12.8525
1.88,-0.128467,-12.8467
1.888,-0.128406,-12.8406
1.896,-0.128341,-12.8341
1.904,-0.128273,-12.8273
1.912,-0.128203,-12.8203
1.92,-0.12813,-12.813
1.928,-0.128056,-12.8056
1.936,-0.12798,-12.798
1.944,-0.127904,-12.7904
1.952,-0.127827,-12.7827
1.96,-0.127751,-12.7751
1.968,-0.127675,-12.7675
1.976,-0.1276,-12.76
1.984,-0.127526,-12.7526
1.992,-0.127453,-12.7453
2,-0.127382,-12.7382
2.008,-0.127314,-12.7314
2.016,-0.127248,-12.7248
2.024,-0.127184,-12.7184
2.032,-0.127123,-12.7123
2.04,-0.127066,-12.7066
2.048,-0.127011,-12.7011
2.056,-0.12696,-12.696
2.064,-0.126913,-12.6913
2.072,-0.126869,-12.6869
2.08,-0.126829,-12.6829
2.088,-0.126793,-12.6793
2.096,-0.12676,-12.676
2.104,-0.126732,-12.6732
2.112,-0.126707,-12.6707
2.12,-0.126686,-12.6686
2.128,-0.126669,-12.6669
2.136,-0.126655,-12.6655
2.144,-0.126646,-12.6646
2.152,-0.126639,-12.6639
2.16,-0.126637,-12.6637
2.168,-0.126637,-12.6637
2.176,-0.126641,-12.6641
2.184,-0.126648,-12.6648
2.192,-0.126658,-12.6658
2.2,-0.12667,-12.667
2.208,-0.126685,-12.6685
2.216,-0.126703,-12.6703
2.224,-0.126723,-12.6723
2.232,-0.126744,-12.6744
2.24,-0.126768,-12.6768
2.248,-0.126793,-12.6793
2.256,-0.12682,-12.682
2.264,-0.126848,-12.6848
2.272,-0.126877,-12.6877
2.28,-0.126907,-12.6907
2.288,-0.126937,-12.6937
2.296,-0.126968,-12.6968
2.304,-0.127,-12.7
2.312,-0.127031,-12.7031
2.32,-0.127063,-12.7063
2.328,-0.127094,-12.7094
2.336,-0.127125,-12.7125
2.344,-0.127156,-12.7156
2.352,-0.127186,-12.7186
2.36,-0.127215,-12.7215
2.368,-0.127243,-12.7243
2.376,-0.127271,-12.7271
2.384,-0.127297,-12.7297
2.392,-0.127322,-12.7322
2.4,-0.127346,-12.7346
2.408,-0.127368,-12.7368
2.416,-0.127389,-12.7389
2.424,-0.127409,-12.7409
2.432,-0.127427,-12.7427
2.44,-0.127444,-12.7444
2.448,-0.127459,-12.7459
2.456,-0.127472,-12.7472
2.464,-0.127484,-12.7484
2.472,-0.127495,-12.7495
2.48,-0.127503,-12.7503
2.488,-0.127511,-12.7511
2.496,-0.127516,-12.7516
2.504,-0.12752,-12.752
2.512,-0.127523,-12.7523
2.52,-0.127524,-12.7524
2.528,-0.127524,-12.7524
2.536,-0.127522,-12.7522
2.544,-0.12752,-12.752
2.552,-0.127516,-12.7516
2.56,-0.127511,-12.7511
2.568,-0.127505,-12.7505
2.576,-0.127497,-12.7497
2.584,-0.127489,-12.7489
2.592,-0.12748,-12.748
2.6,-0.127471,-12.7471
2.608,-0.12746,-12.746
2.616,-0.127449,-12.7449
2.624,-0.127438,-12.7438
2.632,-0.127426,-12.7426
2.64,-0.127414,-12.7414
2.648,-0.127401,-12.7401
2.656,-0.127388,-12.7388
2.664,-0.127375,-12.7375
2.672,-0.127362,-12.7362
2.68,-0.127349,-12.7349
2.688,-0.127336,-12.7336
2.696,-0.127323,-12.7323
2.704,-0.127311,-12.7311
2.712,-0.127298,-12.7298
2.72,-0.127286,-12.7286
2.728,-0.127275,-12.7275
2.736,-0.127263,-12.7263
2.744,-0.127252,-12.7252
2.752,-0.127242,-12.7242
2.76,-0.127232,-12.7232
2.768,-0.127223,-12.7223
2.776,-0.127214,-12.7214
2.784,-0.127206,-12.7206
2.792,-0.127198,-12.7198
2.8,-0.127192,-12.7192
2.808,-0.127185,-12.7185
2.816,-0.12718,-12.718
2.824,-0.127175,-12.7175
2.832,-0.127171,-12.7171
2.84,-0.127167,-12.7167
2.848,-0.127164,-12.7164
2.856,-0.127162,-12.7162
2.864,-0.12716,-12.716
2.872,-0.127159,-12.7159
2.88,-0.127158,-12.7158
2.888,-0.127158,-12.7158
2.896,-0.127159,-12.7159
2.904,-0.12716,-12.716
2.912,-0.127162,-12.7162
2.92,-0.127164,-12.7164
2.928,-0.127166,-12.7166
2.936,-0.127169,-12.7169
2.944,-0.127172,-12.7172
2.952,-0.127176,-12.7176
2.96,-0.12718,-12.718
2.968,-0.127184,-12.7184
2.976,-0.127189,-12.7189
2.984,-0.127194,-12.7194
2.992,-0.127198,-12.7198
3,-0.127204,-12.7204
3.008,-0.127209,-12.7209
3.016,-0.127214,-12.7214
3.024,-0.127219,-12.7219
3.032,-0.127225,-12.7225
3.04,-0.12723,-12.723
3.048,-0.127235,-12.7235
3.056,-0.127241,-12.7241
3.064,-0.127246,-12.7246
3.072,-0.127251,-12.7251
3.08,-0.127256,-12.7256
3.088,-0.127261,-12.7261
3.096,-0.127266,-12.7266
3.104,-0.12727,-12.727
3.112,-0.127274,-12.7274
3.12,-0.127278,-12.7278
3.128,-0.127282,-12.7282
3.136,-0.127286,-12.7286
3.144,-0.127289,-12.7289
3.152,-0.127292,-12.7292
3.16,-0.127295,-12.7295
3.168,-0.127298,-12.7298
3.176,-0.1273,-12.73
3.184,-0.127302,-12.7302
3.192,-0.127304,-12.7304
3.2,-0.127305,-12.7305
3.208,-0.127307,-12.7307
3.216,-0.127308,-12.7308
3.224,-0.127308,-12.7308
3.232,-0.127309,-12.7309
3.24,-0.127309,-12.7309
3.248,-0.127309,-12.7309
3.256,-0.127309,-12.7309
3.264,-0.127308,-12.7308
3.272,-0.127308,-12.7308
3.28,-0.127307,-12.7307
3.288,-0.127306,-12.7306
3.296,-0.127305,-12.7305
3.304,-0.127303,-12.7303
3.312,-0.127302,-12.7302
3.32,-0.1273,-12.73
3.328,-0.127298,-12.7298
3.336,-0.127297,-12.7297
3.344,-0.127295,-12.7295
3.352,-0.127293,-12.7293
3.36,-0.12729,-12.729
3.368,-0.127288,-12.7288
3.376,-0.127286,-12.7286
3.384,-0.127284,-12.7284
3.392,-0.127282,-12.7282
3.4,-0.127279,-12.7279
3.408,-0.127277,-12.7277
3.416,-0.127275,-12.7275
3.424,-0.127273,-12.7273
3.432,-0.127271,-12.7271
3.44,-0.127269,-12.7269
3.448,-0.127267,-12.7267
3.456,-0.127265,-12.7265
3.464,-0.127263,-12.7263
3.472,-0.127261,-12.7261
3.48,-0.127259,-12.7259
3.488,-0.127258,-12.7258
3.496,-0.127256,-12.7256
3.504,-0.127255,-12.7255
3.512,-0.127254,-12.7254
3.52,-0.127252,-12.7252
3.528,-0.127251,-12.7251
3.536,-0.12725,-12.725
3.544,-0.127249,-12.7249
3.552,-0.127249,-12.7249
3.56,-0.127248,-12.7248
3.568,-0.127248,-12.7248
3.576,-0.127247,-12.7247
3.584,-0.127247,-12.7247
3.592,-0.127247,-12.7247
3.6,-0.127247,-12.7247
3.608,-0.127247,-12.7247
3.616,-0.127247,-12.7247
3.624,-0.127247,-12.7247
3.632,-0.127247,-12.7247
3.64,-0.127248,-12.7248
3.648,-0.127248,-12.7248
3.656,-0.127249,-12.7249
3.664,-0.127249,-12.7249
3.672,-0.12725,-12.725
3.68,-0.12725,-12.725
3.688,-0.127251,-12.7251
3.696,-0.127252,-12.7252
3.704,-0.127253,-12.7253
3.712,-0.127254,-12.7254
3.72,-0.127255,-12.7255
3.728,-0.127255,-12.7255
3.736,-0.127256,-12.7256
3.744,-0.127257,-12.7257
3.752,-0.127258,-12.7258
3.76,-0.127259,-12.7259
3.768,-0.12726,-12.726
3.776,-0.127261,-12.7261
3.784,-0.127262,-12.7262
3.792,-0.127263,-12.7263
3.8,-0.127264,-12.7264
3.808,-0.127264,-12.7264
3.816,-0.127265,-12.7265
3.824,-0.127266,-12.7266
3.832,-0.127267,-12.7267
3.84,-0.127267,-12.7267
3.848,-0.127268,-12.7268
3.856,-0.127269,-12.7269
3.864,-0.127269,-12.7269
3.872,-0.12727,-12.727
3.88,-0.12727,-12.727
3.888,-0.127271,-12.7271
3.896,-0.127271,-12.7271
3.904,-0.127271,-12.7271
3.912,-0.127272,-12.7272
3.92,-0.127272,-12.7272
3.928,-0.127272,-12.7272
3.936,-0.127272,-12.7272
3.944,-0.127273,-12.7273
3.952,-0.127273,-12.7273
3.96,-0.127273,-12.7273
3.968,-0.127273,-12.7273
3.976,-0.127273,-12.7273
3.984,-0.127273,-12.7273
3.992,-0.127272,-12.7272
4,-0.127272,-12.7272
4.008,-0.127272,-12.7272
\end{filecontents*}

\begin{filecontents*}{correct.csv}
a,b
0,-0.13
0.008,-0.13
0.016,-0.13
0.024,-0.13
0.032,-0.13
0.04,-0.13
0.048,-0.13
0.056,-0.13
0.064,-0.13
0.072,-0.13
0.08,-0.13
0.088,-0.13
0.096,-0.13
0.104,-0.13
0.112,-0.13
0.12,-0.13
0.128,-0.13
0.136,-0.13
0.144,-0.13
0.152,-0.13
0.16,-0.13
0.168,-0.13
0.176,-0.13
0.184,-0.13
0.192,-0.13
0.2,-0.13
0.208,-0.13
0.216,-0.13
0.224,-0.13
0.232,-0.13
0.24,-0.13
0.248,-0.13
0.256,-0.13
0.264,-0.13
0.272,-0.13
0.28,-0.13
0.288,-0.13
0.296,-0.13
0.304,-0.13
0.312,-0.13
0.32,-0.13
0.328,-0.13
0.336,-0.13
0.344,-0.13
0.352,-0.13
0.36,-0.13
0.368,-0.13
0.376,-0.13
0.384,-0.13
0.392,-0.13
0.4,-0.13
0.408,-0.13
0.416,-0.13
0.424,-0.13
0.432,-0.13
0.44,-0.13
0.448,-0.13
0.456,-0.13
0.464,-0.13
0.472,-0.13
0.48,-0.13
0.488,-0.13
0.496,-0.13
0.504,-0.13
0.512,-0.13
0.52,-0.13
0.528,-0.13
0.536,-0.13
0.544,-0.13
0.552,-0.13
0.56,-0.13
0.568,-0.13
0.576,-0.13
0.584,-0.13
0.592,-0.13
0.6,-0.13
0.608,-0.13
0.616,-0.13
0.624,-0.13
0.632,-0.13
0.64,-0.13
0.648,-0.13
0.656,-0.13
0.664,-0.13
0.672,-0.13
0.68,-0.13
0.688,-0.13
0.696,-0.13
0.704,-0.13
0.712,-0.13
0.72,-0.13
0.728,-0.13
0.736,-0.13
0.744,-0.13
0.752,-0.13
0.76,-0.13
0.768,-0.13
0.776,-0.13
0.784,-0.13
0.792,-0.13
0.8,-0.13
0.808,-0.13
0.816,-0.13
0.824,-0.13
0.832,-0.13
0.84,-0.13
0.848,-0.13
0.856,-0.13
0.864,-0.13
0.872,-0.13
0.88,-0.13
0.888,-0.13
0.896,-0.13
0.904,-0.13
0.912,-0.13
0.92,-0.13
0.928,-0.13
0.936,-0.13
0.944,-0.13
0.952,-0.13
0.96,-0.13
0.968,-0.13
0.976,-0.13
0.984,-0.13
0.992,-0.13
1,-0.13
1.008,-0.13
1.016,-0.13
1.024,-0.13
1.032,-0.13
1.04,-0.13
1.048,-0.13
1.056,-0.13
1.064,-0.13
1.072,-0.13
1.08,-0.13
1.088,-0.13
1.096,-0.13
1.104,-0.13
1.112,-0.13
1.12,-0.13
1.128,-0.13
1.136,-0.13
1.144,-0.13
1.152,-0.13
1.16,-0.13
1.168,-0.13
1.176,-0.13
1.184,-0.13
1.192,-0.13
1.2,-0.13
1.208,-0.13
1.216,-0.13
1.224,-0.13
1.232,-0.13
1.24,-0.13
1.248,-0.13
1.256,-0.13
1.264,-0.13
1.272,-0.13
1.28,-0.13
1.288,-0.13
1.296,-0.13
1.304,-0.13
1.312,-0.13
1.32,-0.13
1.328,-0.13
1.336,-0.13
1.344,-0.13
1.352,-0.13
1.36,-0.13
1.368,-0.13
1.376,-0.13
1.384,-0.13
1.392,-0.13
1.4,-0.13
1.408,-0.13
1.416,-0.13
1.424,-0.13
1.432,-0.13
1.44,-0.13
1.448,-0.13
1.456,-0.13
1.464,-0.13
1.472,-0.13
1.48,-0.13
1.488,-0.13
1.496,-0.13
1.504,-0.13
1.512,-0.13
1.52,-0.13
1.528,-0.13
1.536,-0.13
1.544,-0.13
1.552,-0.13
1.56,-0.13
1.568,-0.13
1.576,-0.13
1.584,-0.13
1.592,-0.13
1.6,-0.13
1.608,-0.13
1.616,-0.13
1.624,-0.13
1.632,-0.13
1.64,-0.13
1.648,-0.13
1.656,-0.13
1.664,-0.13
1.672,-0.13
1.68,-0.13
1.688,-0.13
1.696,-0.13
1.704,-0.13
1.712,-0.13
1.72,-0.13
1.728,-0.13
1.736,-0.13
1.744,-0.13
1.752,-0.13
1.76,-0.13
1.768,-0.13
1.776,-0.13
1.784,-0.13
1.792,-0.13
1.8,-0.13
1.808,-0.13
1.816,-0.13
1.824,-0.13
1.832,-0.13
1.84,-0.13
1.848,-0.13
1.856,-0.13
1.864,-0.13
1.872,-0.13
1.88,-0.13
1.888,-0.13
1.896,-0.13
1.904,-0.13
1.912,-0.13
1.92,-0.13
1.928,-0.13
1.936,-0.13
1.944,-0.13
1.952,-0.13
1.96,-0.13
1.968,-0.13
1.976,-0.13
1.984,-0.13
1.992,-0.13
2,-0.13
2.008,-0.13
2.016,-0.13
2.024,-0.13
2.032,-0.13
2.04,-0.13
2.048,-0.13
2.056,-0.13
2.064,-0.13
2.072,-0.13
2.08,-0.13
2.088,-0.13
2.096,-0.13
2.104,-0.13
2.112,-0.13
2.12,-0.13
2.128,-0.13
2.136,-0.13
2.144,-0.13
2.152,-0.13
2.16,-0.13
2.168,-0.13
2.176,-0.13
2.184,-0.13
2.192,-0.13
2.2,-0.13
2.208,-0.13
2.216,-0.13
2.224,-0.13
2.232,-0.13
2.24,-0.13
2.248,-0.13
2.256,-0.13
2.264,-0.13
2.272,-0.13
2.28,-0.13
2.288,-0.13
2.296,-0.13
2.304,-0.13
2.312,-0.13
2.32,-0.13
2.328,-0.13
2.336,-0.13
2.344,-0.13
2.352,-0.13
2.36,-0.13
2.368,-0.13
2.376,-0.13
2.384,-0.13
2.392,-0.13
2.4,-0.13
2.408,-0.13
2.416,-0.13
2.424,-0.13
2.432,-0.13
2.44,-0.13
2.448,-0.13
2.456,-0.13
2.464,-0.13
2.472,-0.13
2.48,-0.13
2.488,-0.13
2.496,-0.13
2.504,-0.13
2.512,-0.13
2.52,-0.13
2.528,-0.13
2.536,-0.13
2.544,-0.13
2.552,-0.13
2.56,-0.13
2.568,-0.13
2.576,-0.13
2.584,-0.13
2.592,-0.13
2.6,-0.13
2.608,-0.13
2.616,-0.13
2.624,-0.13
2.632,-0.13
2.64,-0.13
2.648,-0.13
2.656,-0.13
2.664,-0.13
2.672,-0.13
2.68,-0.13
2.688,-0.13
2.696,-0.13
2.704,-0.13
2.712,-0.13
2.72,-0.13
2.728,-0.13
2.736,-0.13
2.744,-0.13
2.752,-0.13
2.76,-0.13
2.768,-0.13
2.776,-0.13
2.784,-0.13
2.792,-0.13
2.8,-0.13
2.808,-0.13
2.816,-0.13
2.824,-0.13
2.832,-0.13
2.84,-0.13
2.848,-0.13
2.856,-0.13
2.864,-0.13
2.872,-0.13
2.88,-0.13
2.888,-0.13
2.896,-0.13
2.904,-0.13
2.912,-0.13
2.92,-0.13
2.928,-0.13
2.936,-0.13
2.944,-0.13
2.952,-0.13
2.96,-0.13
2.968,-0.13
2.976,-0.13
2.984,-0.13
2.992,-0.13
3,-0.13
3.008,-0.13
3.016,-0.13
3.024,-0.13
3.032,-0.13
3.04,-0.13
3.048,-0.13
3.056,-0.13
3.064,-0.13
3.072,-0.13
3.08,-0.13
3.088,-0.13
3.096,-0.13
3.104,-0.13
3.112,-0.13
3.12,-0.13
3.128,-0.13
3.136,-0.13
3.144,-0.13
3.152,-0.13
3.16,-0.13
3.168,-0.13
3.176,-0.13
3.184,-0.13
3.192,-0.13
3.2,-0.13
3.208,-0.13
3.216,-0.13
3.224,-0.13
3.232,-0.13
3.24,-0.13
3.248,-0.13
3.256,-0.13
3.264,-0.13
3.272,-0.13
3.28,-0.13
3.288,-0.13
3.296,-0.13
3.304,-0.13
3.312,-0.13
3.32,-0.13
3.328,-0.13
3.336,-0.13
3.344,-0.13
3.352,-0.13
3.36,-0.13
3.368,-0.13
3.376,-0.13
3.384,-0.13
3.392,-0.13
3.4,-0.13
3.408,-0.13
3.416,-0.13
3.424,-0.13
3.432,-0.13
3.44,-0.13
3.448,-0.13
3.456,-0.13
3.464,-0.13
3.472,-0.13
3.48,-0.13
3.488,-0.13
3.496,-0.13
3.504,-0.13
3.512,-0.13
3.52,-0.13
3.528,-0.13
3.536,-0.13
3.544,-0.13
3.552,-0.13
3.56,-0.13
3.568,-0.13
3.576,-0.13
3.584,-0.13
3.592,-0.13
3.6,-0.13
3.608,-0.13
3.616,-0.13
3.624,-0.13
3.632,-0.13
3.64,-0.13
3.648,-0.13
3.656,-0.13
3.664,-0.13
3.672,-0.13
3.68,-0.13
3.688,-0.13
3.696,-0.13
3.704,-0.13
3.712,-0.13
3.72,-0.13
3.728,-0.13
3.736,-0.13
3.744,-0.13
3.752,-0.13
3.76,-0.13
3.768,-0.13
3.776,-0.13
3.784,-0.13
3.792,-0.13
3.8,-0.13
3.808,-0.13
3.816,-0.13
3.824,-0.13
3.832,-0.13
3.84,-0.13
3.848,-0.13
3.856,-0.13
3.864,-0.13
3.872,-0.13
3.88,-0.13
3.888,-0.13
3.896,-0.13
3.904,-0.13
3.912,-0.13
3.92,-0.13
3.928,-0.13
3.936,-0.13
3.944,-0.13
3.952,-0.13
3.96,-0.13
3.968,-0.13
3.976,-0.13
3.984,-0.13
3.992,-0.13
4,-0.13
4.008,-0.13
\end{filecontents*}

\urlstyle{same}
\author[Thrane Hansen, Simon] % (optional)
{S. Thrane Hansen, C. Thule and C. Gomes}
\institute[AU] % (optional)
{
  Department of Engineering\\
  %Computer Engineering \\
  Aarhus University
}

\date[AU 2020] % (optional)
{Presentation CO-sim CPS, \today}

%\logo{\includegraphics[height=1.5cm]{lion-logo.jpg}}

%End of title page configuration block
%------------------------------------------------------------



%------------------------------------------------------------
%The next block of commands puts the table of contents at the 
%beginning of each section and highlights the current section:


%------------------------------------------------------------
\begin{document}

%The next statement creates the title page.
\frame{\titlepage}

%---------------------------------------------------------
%This block of code is for the table of contents after
%the title page
\begin{frame}
\frametitle{Table of Contents}
\tableofcontents
\end{frame}
%---------------------------------------------------------

\section{The Initialization Problem}
\begin{frame}
\frametitle{Developing Cyber-Physical Systems are hard!}
\begin{itemize}
    \item The problems the systems should solve are hard
    \item The design of the system is complex
    \item The systems consist of heterogeneous parts
    \item The development cycles are getting shorter and shorter
\end{itemize}

\end{frame}

\begin{frame}
\frametitle{Co-Simulation is a tool to analyze \textbf{non-trivial} systems}
\begin{itemize}
    \item FMU-based co-simulation is one way to handle the complexity of these systems
    \item Co-simulation give us a way to analyze and reason about CPS.
    \item The Functional Mock-Up Interface handles the different types of the simulation units
\end{itemize}

\end{frame}

\begin{frame}
\frametitle{FMU-based Co-Simulation}
\begin{columns}
\column{0.5\textwidth}
\begin{itemize}
    \item 3 phases (Initialization, Simulation, Teardown)
    \item FMI describes the interface of each component
    \item The simulation is controlled by an Master Algorithm
\end{itemize}
\column{0.5\textwidth}
\begin{figure}
    \centering
    \includegraphics[scale=0.3]{images/Screenshot 2020-09-09 at 08.38.44.png}
\end{figure}
\end{columns}

The goal of the \textbf{Initialization} phase is define all FMU variables and assign them a \textbf{stable} initial value.
\end{frame}

\begin{frame}
\frametitle{But doesn't FMI describe the Initialization of a Co-Simulation?}
FMI describes how to initialize a single FMU and non-connected FMU variables.
The initialization of connected FMUs introduces more challenges (precedence constraints).
\textbf{Interconnected FMU variables require a specific initialization order}.

\begin{figure}
    \centering
    \includegraphics[scale=1.5]{images/ExpansionPlugin-Page-3.pdf}
\end{figure}
Initialization order: $ct.Valve -> wt.valveControl -> wt.level -> ct.level$
\begin{alertblock}{Initialization order} 
    The initialization order is not always trivial! 
\end{alertblock}
\end{frame}

\begin{frame}
\frametitle{Example of a non-trivial Co-Simulation scenario}
A simplified version of a car's suspension system.
\begin{figure}
    \centering
    \includegraphics[scale=0.6]{images/quarter_car.pdf}
    \label{fig:SCC_quarter}
\end{figure}
\end{frame}

\begin{frame}
\frametitle{Non-trivial Co-Simulation scenarios}
\begin{figure}
    \centering
    \includegraphics[scale=0.6]{images/quarter_car_SCC.pdf}
    \label{fig:SCC_quarter}
\end{figure}
\begin{itemize}
    \item Algebraic loop - an FMU variable dependence on itself.
\end{itemize}
A loop needs a specific initialization strategy to achieve a correct simulation result!
\textbf{Not} all scenarios containing are stable.
\end{frame}

\begin{frame}[fragile]
\frametitle{Importance of initializing the loops correctly}
\begin{figure}
\centering
\begin{tikzpicture}
    \begin{axis}[
        at={(0,0)},
        width= 10cm,
        height = 7cm,
        ymin=-20,
        ymax=5,
        xmin=0,
        xmax=4.1,
        grid=both,
        grid style={line width=.1pt, draw=gray!10},
        major grid style={line width=.2pt,draw=gray!50},
        axis lines=middle,
        minor tick num =5,
        axis line style={latex-latex},
        ticklabel style={font=\tiny,fill=white},
        minor tick style={draw=none},
        minor grid style={thin,color=black!10},
        ylabel= Car chassis position (cm),
        xlabel= Time (s),
        tick align=outside,
        x label style={at={(axis description cs:0.5,-0.1)},anchor=north, color=blue!50!cyan},
        y label style={at={(axis description cs:-0.1,.5)},rotate=90,anchor=south, color=blue!50!cyan},
        x tick label style={
            /pgf/number format/assume math mode, font=\sf\scriptsize}
        ]
    \addplot [line width=2.5pt, red] table [x= a, y=c, col sep=comma] {incorrect.csv};
    \addplot [y filter/.code={\pgfmathparse{\pgfmathresult*100.}\pgfmathresult}][line width=2.5pt, blue] table [x=a, y=b, col sep=comma] {correct.csv};
\end{axis}
\end{tikzpicture}
    \caption{Simulation results}
    \label{fig:init_state_0_sim}
\end{figure}

\end{frame}

\section{So how to handle these non-trivial scenarios?}
\begin{frame}
\frametitle{So how to handle do the obtain a valid initialization of a non-trivial scenario?}
The recipe:
\begin{enumerate}[I]
    \item Identify the characteristics of the co-simulation scenario
    \begin{itemize}
        \item Non-trivial - algebraic loops
        \item Trivial - no algebraic loops
    \end{itemize}
    \item Calculation of a correct initialization order
    \item Initializing the interconnected FMU-variables
\end{enumerate}

\textbf{The characteristics of the co-simulation scenario determines the initialization strategy}
\end{frame}

\begin{frame}
\frametitle{Identification of non-trivial co-simulation scenarios}
\begin{itemize}
    \item A graph-based a approach 
    \item The problem is to find all SCCs of the graph
\end{itemize}
\begin{figure}
    \centering
    \includegraphics[width=0.9\textwidth]{images/quarter_car-graph.pdf}
    \caption{Generation of a graph of interconnected FMU-variables}
\end{figure}
\end{frame}

\begin{frame}
\frametitle{Identification of SCCs in the graph}
A simple matter of picking a graph algorithm that solves this problem.
\begin{figure}
    \centering
    \includegraphics[width=0.6\textwidth]{images/quarter_car-SCC.pdf}
\end{figure}

\end{frame}

\begin{frame}
\frametitle{Calculation of a correct initialization order}
The initialization order is then the topological ordering of the SCCs.

\begin{exampleblock}{Optimization of the initialization order}
Variables of a single FMU can be set/read in bulk!
\end{exampleblock}
The initialization order can often be optimized!
\end{frame}

\begin{frame}
\frametitle{Reminder of the recipe}
\begin{enumerate}[I]
    \item Identify the characteristics of the co-simulation scenario
    \begin{itemize}
        \item Non-trivial - algebraic loops
        \item Trivial - no algebraic loops
    \end{itemize}
    \item Calculation of a correct initialization order
    \item Initializing the interconnected FMU-variables using the right strategy in the right order
    \begin{itemize}
        \item Simple FMU-action
        \item Fixed-point iteration
    \end{itemize}
\end{enumerate}
\end{frame}

\begin{frame}
\frametitle{The initialization strategy for trivial SCC}
The initialization of trivial SCC is performing the correspondent variable action (set,get) 
\begin{figure}
    \centering
    \includegraphics[scale=1.5]{images/ExpansionPlugin-Page-3.pdf}
\end{figure}
\end{frame}
        
\begin{frame}
\frametitle{The initialization strategy for algebraic loops}
Each loop is initialized using a fixed point iteration strategy until convergence.
    
\begin{figure}
    \centering
    \includegraphics[scale=0.6]{images/ExpansionPlugin-Page-4.pdf}
\end{figure}
    
\begin{alertblock}{Convergence}
    The convergence of a loop is not guaranteed!
\end{alertblock}
    
\end{frame}

\begin{frame}
\frametitle{Checking for convergence}
How do we make sure that our system has reached a stable initial state?
Within a finite number of iterations convergence should  be established.
\begin{figure}
    \centering
    \includegraphics[scale=0.3]{images/Screenshot 2020-09-09 at 20.51.28.png}
\end{figure}

If convergence is not established, the co-simulation is dismissed!
\end{frame}



\begin{frame}[fragile]
\frametitle{The complete approach}
\begin{figure}
    \centering
    \includegraphics[scale=0.25]{images/Screenshot 2020-09-09 at 09.12.56.png}
\end{figure}

\begin{block}{Remark} 
    The approach is suitable with all well-established master algorithms ()
\end{block}
\end{frame}


\begin{frame}
\frametitle{INTO-CPS Maestro 2}
\begin{itemize}
    \item A part of the INTO-CPS application and a successor of the Maestro.    
    \item Separates the specification (MaBL) of a co-simulation from the simulation.
    \item The specification is typically generated by some plugins
\end{itemize}
\begin{figure}
    \centering
    \includegraphics[scale=0.6]{images/ExpansionPlugin-Page-1.pdf}
    \label{fig:my_label}
\end{figure}
\end{frame}


\begin{frame}[fragile]
\frametitle{Maestro Base Language (MaBL)}
The specification of a co-simulation is expressed in MaBL.

\begin{lstlisting}[language=C++]
simulation
import Initializer;
{
FMI2 tankcontroller = load("FMI2", "{8c4e810f-3df3-4a00-8276-176fa3c9f000}", "src/test/resources/watertankcontroller-c.fmu");
...
FMI2Component crtlInstance = tankcontroller.instantiate("crtlInstance", false, false);;
IFmuComponent components[2]={wtInstance,crtlInstance};
expand initialize(components,START_TIME, END_TIME);

SingleWatertank.freeInstance(wtInstance);
}
\end{lstlisting}
\end{frame}


%---------------------------------------------------------
%Changing visibility of the text
\begin{frame}
\frametitle{The implementation of the approach}
\begin{itemize}
    \item An expansion plugin to Maestro 2
    \item The plugin generates the initialization procedure
    \item The plugin can handle both trivial and non-trivial co-simulation scenarios
    \item The identification of SCCs and topological sorting is performed by Tarjan's Algorithm implemented in Scala
    \item The expansion plugin can be used with an arbitrary MA
    \item The plugin can be configured to define the criteria of convergence and other parameters of the initialization.
\end{itemize}
\end{frame}



\begin{frame}
\frametitle{Verification of the plugin}
\begin{itemize}
    \item The plugin has been verified against the existing \textit{Initializer}
    \item Gomes et al. made an encoding of the rules of order between the FMU operations in Prolog
    \item The calculated initialization order is verified against co-simulation step verifier in Prolog.
\end{itemize}
\end{frame}


\begin{frame}
\frametitle{Summary - the key takeaways!}
\begin{itemize}
    \item A correct initialization of a co-simulation is crucial to obtain a trustworthy result!
    \item The topological ordering of the interconnected FMU variables is the initialization order
    \item Co-simulation scenarios containing algebraic loops should be initialized using a fixed-point strategy
    \item The approach is implemented as an expansion plugin to Maestro 2  
\end{itemize}
\end{frame}

\section{Future work}

\begin{frame}
\frametitle{Future work}
\begin{itemize}
    \item Formally verification the Expansion Plugin  
    \begin{itemize}
        \item Using the Sireum framework
    \end{itemize}
    \item Formal verification to guarantee termination of a co-simulation in Maestro 2.
\end{itemize}
\end{frame}

\begin{frame}
\frametitle{Questions}
\huge
Thank you!
\end{frame}


\end{document}